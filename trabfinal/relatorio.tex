% ------------------------------------------------------------- 
% Arquivo :  relatório modelo                                        
% ------------------------------------------------------------- 
% O percentual(%) serve para incluir comentários: 
% tudo o que fica à direita dele não é interpretado pelo LaTex
% Linhas e espaços em branco também **NÃO** são
% interpretadas pelo LaTex

%% As intruções seguintes são o cabeçalho e devem estar antes do
%% \begin{document}

%\documenclass: mandatorio, indica o tipo/formato de documento
\documentclass[brazilian,12pt,a4paper,final]{article}
% tamanhos de fontes: 10pt, 11pt ou 12pt
% opções de estilo (padrões): article, report, book, slide, letter (artigo, relatorio, livro, apresentação de slides, carta)




%% Pacotes extras (opcionais):

% *babel* contem as regras de ifenização
\usepackage[brazil]{babel}

% *t1enc* permite o reconhecimento dos acentos inseridos com o teclado
%\usepackage{t1enc}

% *inputenc* com opção *utf8* permite reconhecimento dos caracteres com codificação UTF8, que é padrão dos esditores de texto no Linux. Isso permite reconhecimento automático de acentuação.
\usepackage[utf8]{inputenc}


% *graphicx* é para incluir figuras em formato eps 
\usepackage{graphicx} % para produzir PDF diretamente reescrever esta linha assim: \usepackage[pdftex]{graphicx}

% *color* fontes soloridas
\usepackage{color}
%%% fim do cabecalho %%%

\pagestyle{empty}
\title{Métodos Computacionais da Física B \\ Amortecimento devido à resistência do ar}
\author{Aluno: Átila Leites Romero - Matrícula: 144679 \\ IF-UFRGS}

\begin{document}
\maketitle
%\begin{abstract}
%Descrever de forma sintética o problema 
%e os resultados.
% as quebras de linha coma a de cima não são interpretadas
% para forzar quebra de linha deve ser deixada uma linha em branco
%\end{abstract}

%Abaixo podem ver como se deve colocar letras acentuadas ou latinas se
%o pacote *t1enc* não dfosse usado 
\section{Introdução} 
% Aqui a Introdução \c{c} e \~a  é a forma standar  de escrever
% carateres ASCII extendidos (acentos, etc), porem com o pacote t1enc
% declarado acima podemos escrever diretamente ç em lugar d \c{c}, etc
É comum que o amortecimento de um corpo dependa linearmente ou quadraticamente da velocidade. 
Existem situações em que o amortecimento depende simultâneamente destes dois fatores, o que dificulta a obtenção de uma solução analítica.

Um exemplo de sistema físico onde isto ocorre é no amortecimento devido à resistência do ar. Para baixas velocidades, a dependência linear predomina enquanto para altas velocidades predomina a dependência quadrática.

Em algumas situações, considera-se apenas o fator predominante. No entanto, esta aproximação pode não fornecer resultados precisos em situações intermediárias. Além disto, é difícil tratar analiticamente a parte quadrática, mesmo quando isoladamente.

O amortecimento em baixas velocidades pode ser descrito por\cite{stokes_wiki}:
$$F_V=-C_V v$$
onde $F_V$ é a força de atrito devido à viscosidade, $C_V$ é um coeficiente que depende da viscosidade do fluido e das dimensões do corpo e $v$ é a velocidade do corpo. Para pequenos objetos esféricos, $b=6\pi \eta r$, onde $\eta$ é o coeficiente de viscosidade e $r$ o raio.

O amortecimento em altas velocidades pode ser descrito por\cite{drag1_wiki}:
$$F_D=-\frac{1}{2}\rho A C_D v^2$$
onde $F_D$ é o arrasto quadrático, $\rho$ é a densidade do ar, $A$ é a área de referência, $C_D$ é o coeficiente de arrasto e $v$ a velocidade do corpo.

\section{Método}
% Aqui o Método 
%Detalhes sobre o método utilizado ,
%procure ao final do texto a referencia a esta bibliografia.
%demonstrações de porque ele funciona.  Limites analíticos, etc.


\section{Resultados}
 Aqui os resultados, sua interpretação.

% Aqui incluimos uma figura
% Para testar este comando devem criar uma figura em formato
% postscript encapsulado (eps) e deve ter o mesmo nome que aparece entre chaves
% no caso ``fig.eps'' 

\section{Conclusões}
Recolocar resumidamente o problema, os resultados, as comparações
trabalhos e as perspectivas futuras que o trabalho abre.

{\color{red} Este é um modelo geral, quando for utilizá-lo para um trabalho
específico leve em consideração as necesidades desse trabalho,
cuidando de omitir ou comentar com \% \% as seções que não
se apliquem.}

\begin{thebibliography}{99}

\bibitem{drag1_wiki}
Drag (physics), 
Disponível em: http://en.wikipedia.org/wiki/Drag\_(physics).
Acesso em 03/07/2012.

\bibitem{stokes_wiki}
Stokes' law, 
Disponível em: http://en.wikipedia.org/wiki/Stokes'\_law.
Acesso em 03/07/2012.

\end{thebibliography}

\end{document}

