% ------------------------------------------------------------- 
% Arquivo :  relatório modelo                                        
% ------------------------------------------------------------- 
% O percentual(%) serve para incluir comentários: 
% tudo o que fica à direita dele não é interpretado pelo LaTex
% Linhas e espaços em branco também **NÃO** são
% interpretadas pelo LaTex

%% As intruções seguintes são o cabeçalho e devem estar antes do
%% \begin{document}

%\documenclass: mandatorio, indica o tipo/formato de documento
\documentclass[brazilian,12pt,a4paper,final]{article}
% tamanhos de fontes: 10pt, 11pt ou 12pt
% opções de estilo (padrões): article, report, book, slide, letter (artigo, relatorio, livro, apresentação de slides, carta)




%% Pacotes extras (opcionais):

% *babel* contem as regras de ifenização
\usepackage[brazil]{babel}

% *t1enc* permite o reconhecimento dos acentos inseridos com o teclado
%\usepackage{t1enc}

% *inputenc* com opção *utf8* permite reconhecimento dos caracteres com codificação UTF8, que é padrão dos esditores de texto no Linux. Isso permite reconhecimento automático de acentuação.
\usepackage[utf8]{inputenc}


% *graphicx* é para incluir figuras em formato eps 
\usepackage{graphicx} % para produzir PDF diretamente reescrever esta linha assim: \usepackage[pdftex]{graphicx}

% *color* fontes soloridas
\usepackage{color}
%%% fim do cabecalho %%%

\pagestyle{empty}
\title{Métodos Computacionais da Física B \\ Amortecimento em um fluido}
\author{Aluno: Átila Leites Romero - Matrícula: 144679 \\ IF-UFRGS}

\begin{document}
\maketitle
%\begin{abstract}
%Descrever de forma sintética o problema 
%e os resultados.
% as quebras de linha coma a de cima não são interpretadas
% para forzar quebra de linha deve ser deixada uma linha em branco
%\end{abstract}

%Abaixo podem ver como se deve colocar letras acentuadas ou latinas se
%o pacote *t1enc* não dfosse usado 
\section{Introdução} 
% Aqui a Introdução \c{c} e \~a  é a forma standar  de escrever
% carateres ASCII extendidos (acentos, etc), porem com o pacote t1enc
% declarado acima podemos escrever diretamente ç em lugar d \c{c}, etc
Neste trabalho, são comparados dois modelos para o amortecimento de um corpo em movimento em um fluido, ambos dependentes da velocidade, porém um dependentee linearmente e o outro quadraticamente. Os modelos também podem ser combinados em um único.

Um exemplo de sistema físico onde isto ocorre é no amortecimento devido à resistência do ar. Para baixas velocidades, a dependência linear predomina enquanto para altas velocidades predomina a dependência quadrática.

Em algumas situações, considera-se apenas o fator predominante. No entanto, esta aproximação pode não fornecer resultados precisos em situações intermediárias. Além disto, é difícil tratar analiticamente a parte quadrática, mesmo quando isoladamente.

O amortecimento em baixas velocidades pode ser descrito por\cite{stokes_wiki}:
$$F_V=-C_V v$$
onde $F_V$ é a força de atrito devido à viscosidade, $C_V$ é um coeficiente que depende da viscosidade do fluido e das dimensões do corpo e $v$ é a velocidade do corpo. Para pequenos objetos esféricos, $b=6\pi \eta r$, onde $\eta$ é o coeficiente de viscosidade e $r$ o raio.

O amortecimento em altas velocidades pode ser descrito por\cite{drag1_wiki}:
$$F_D=-\frac{1}{2}\rho A C_D v^2$$
onde $F_D$ é o arrasto quadrático, $\rho$ é a densidade do fluido, $A$ é a área de referência, $C_D$ é o coeficiente de arrasto e $v$ a velocidade do corpo.

\section{Método}
Para comparar estes dois tipos de atrito, foi utilizado um modelo em uma dimensão, em que um corpo é arremessado para cima imerso em um fluido hipotético.

Utilizando gráficos $x\times t$, podem ser comparados os tempos de subida e descida dos modelos.

Para o cálculo de cada percurso, 
foi utilizado o algoritmo de Runge-Kutta de 4a ordem, considerando
$$F_{resultante}=F_V+F_D+F_peso$$

\section{Resultados}
Com baixa velocidade e coeficientes 

Inicialmente, foram comparados os resultados de cada modelo para três diferentes velocidades (figuras
\ref{v2-visc0.1-quad0.1.eps},
\ref{v10-visc0.1-quad0.1.eps} e
\ref{v100-visc0.1-quad0.1.eps}).

Como esperado, quanto mais alta a velocidade, maior a influência da dependência quadrática da velocidade.
Nota-se também que desprezar a influência da viscosidade pode fornecer resultados imprecisos, visto que nos dois primeiros gráficos ela não é desprezível.

\ref{v10-visc0.1-quad0.2.eps}
\ref{v10-visc0.2-quad0.1.eps}

\begin{figure}[htbp!]
  \caption{Velocidade baixa \\ v=2 visc=0.1 quad=0.1}
  \label{v2-visc0.1-quad0.1.eps}
  \centering
    \resizebox{8.0cm}{!}{\includegraphics{v2-visc0.1-quad0.1.eps}}
\end{figure}

\begin{figure}[htbp!]
  \caption{v=10 visc=0.1 quad=0.1}
  \label{v10-visc0.1-quad0.1.eps}
  \centering
    \resizebox{8.0cm}{!}{\includegraphics{v10-visc0.1-quad0.1.eps}}
\end{figure}

\begin{figure}[htbp!]
  \caption{v=100 visc=0.1 quad=0.1}
  \label{v100-visc0.1-quad0.1.eps}
  \centering
    \resizebox{8.0cm}{!}{\includegraphics{v100-visc0.1-quad0.1.eps}}
\end{figure}

\begin{figure}[htbp!]
  \caption{v10 visc=0.1 quad=0.2}
  \label{v10-visc0.1-quad0.2.eps}
  \centering
    \resizebox{8.0cm}{!}{\includegraphics{v10-visc0.1-quad0.2.eps}}
\end{figure}

\begin{figure}[htbp!]
  \caption{v=10 visc=0.2 quad=0.1}
  \label{v10-visc0.2-quad0.1.eps}
  \centering
    \resizebox{8.0cm}{!}{\includegraphics{v10-visc0.2-quad0.1.eps}}
\end{figure}

% Aqui incluimos uma figura
% Para testar este comando devem criar uma figura em formato
% postscript encapsulado (eps) e deve ter o mesmo nome que aparece entre chaves
% no caso ``fig.eps'' 

\section{Conclusões}
Recolocar resumidamente o problema, os resultados, as comparações
trabalhos e as perspectivas futuras que o trabalho abre.

{\color{red} Este é um modelo geral, quando for utilizá-lo para um trabalho
específico leve em consideração as necesidades desse trabalho,
cuidando de omitir ou comentar com \% \% as seções que não
se apliquem.}

\begin{thebibliography}{99}

\bibitem{drag1_wiki}
Drag (physics), 
Disponível em: http://en.wikipedia.org/wiki/Drag\_(physics).
Acesso em 03/07/2012.

\bibitem{stokes_wiki}
Stokes' law, 
Disponível em: http://en.wikipedia.org/wiki/Stokes'\_law.
Acesso em 03/07/2012.

\end{thebibliography}

\end{document}

